

\documentclass[11pt,psfig]{article}
\usepackage{epsfig}
\usepackage{times}
\usepackage{amssymb}
\usepackage{float}

\newcount\refno\refno=1
\def\ref{\the\refno \global\advance\refno by 1}
\def\ux{\underline{x}}
\def\ut{\underline{\theta}}
\def\umu{\underline{\mu}}
\def\be{p_e^*}
\newcount\eqnumber\eqnumber=1
\def\eq{\the \eqnumber \global\advance\eqnumber by 1}
\def\eqs{\eq}
\def\eqn{\eqno(\eq)}

 \pagestyle{empty}
\def\baselinestretch{1.1}
\topmargin1in \headsep0.3in
\topmargin0in \oddsidemargin0in \textwidth6.5in \textheight8.5in
\begin{document}
\setlength{\parskip}{1.2ex plus0.3ex minus 0.3ex}

\subsection*{Proof of Orthogonality}

To prove that our matrix is orthogonal, we need to prove that any two vectors have dot product of zero. In this case the number of data points is $n$ and for simplicity of the proof we will let $N = n-1$. 
\\
Let $\phi = \frac{2\pi}{N}$
\\
We will have to prove that
\[
\sum_{k=0}^{N/2}{cos(t \cdot k \cdot \phi)cos(t' \cdot k \cdot \phi) + sin(t \cdot k \cdot \phi)sin(t' \cdot k \cdot \phi)} = 0
\]
By Trig Identities it holds that
\[
cos(A-B) = cos(A)cos(B) + sin(A)sin(B)
\]
We can then assert that the above equation is equivalent to saying
\[
\sum_{k=0}^{N/2}{cos((t-t') \cdot k \cdot \phi)} = 0
\]
Let $C = |t-t'|$. Since $cos(\theta)=cos(-\theta)$ we can assume that $t-t'>0$. Thus we just have to prove that
\[
\sum_{k=0}^{N/2}{cos(\frac{2\pi C k}{N})} = 0
\]
For each angle, there is another one on the other side of $\pi/2$. **INSERT DETAILS**


\end{document}



