

\documentclass[11pt,psfig]{article}
\usepackage{epsfig}
\usepackage{times}
\usepackage{amssymb}
\usepackage{float}

\newcount\refno\refno=1
\def\ref{\the\refno \global\advance\refno by 1}
\def\ux{\underline{x}}
\def\ut{\underline{\theta}}
\def\umu{\underline{\mu}}
\def\be{p_e^*}
\newcount\eqnumber\eqnumber=1
\def\eq{\the \eqnumber \global\advance\eqnumber by 1}
\def\eqs{\eq}
\def\eqn{\eqno(\eq)}

 \pagestyle{empty}
\def\baselinestretch{1.1}
\topmargin1in \headsep0.3in
\topmargin0in \oddsidemargin0in \textwidth6.5in \textheight8.5in
\begin{document}
\setlength{\parskip}{1.2ex plus0.3ex minus 0.3ex}

\subsection*{Convolution Theorem for our example}

Let the following be true
\[
T_1(f) = \int{f(x) cos(\phi x) \,dx}
\]
\[
T_2(f) = \int{f(x) sin(\phi x) \,dx}
\]
Take the functions $f$ and $g$ and let $h = f*g$, then it is conjectured that
\[
T_1(h) = T_1(f) T_1(g) - T_2(f)T_2(g)
\]
\[
T_2(h) = T_1(f) T_2(g) + T_1(g)T_2(f)
\]
This comes from the fact that in the proof of the convolution theorem, when computing the integral for the convolution, they are able to separate the integral due to the fact that $e^{x+y} = e^x  e^y$. In our case,
\[
cos( \phi(x+y) ) = cos(\phi x)cos(\phi y) - sin(\phi x)sin(\phi y)
\]
\[
sin( \phi(x+y) ) = cos(\phi x)sin(\phi y) + sin(\phi x)cos(\phi y)
\]
Applying the above identities to the integral for $T_1$ and $T_2$ should lead to the conjecture. 
\\

\subsection*{Proving Convolution Theorem in Normal Case}
With convolution theorem, it says that
\[
F(f*g) = F(f) \cdot F(g)
\]
The point-wise multiplication means multiplication of the frequencies. \\

Here is the procedure for doing it in Matlab
\begin{itemize}
\item The variables data,filter will be used
\item Let filteredData = conv(data,filter)
\item Let paddedData and paddedFilter be data and filter but with zeros added to the end
\item Do fft(filteredData) and notice that it is equal to fft(paddedData).*fft(paddedFilter)
\end{itemize}

\subsection*{Possible proof in our case}

We will repeat the same procedure except that for the last item, we will do our own transform on the three arrays. 

\end{document}



